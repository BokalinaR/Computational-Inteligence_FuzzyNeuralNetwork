% !TEX encoding = UTF-8 Unicode
\documentclass[a4paper]{article}
\usepackage{geometry}
 \geometry{
 a4paper,
 total={150mm,240mm},
 left=30mm,
 top=30mm,
 }

\usepackage{color}
\usepackage{url}
\usepackage[T2A]{fontenc} % enable Cyrillic fonts
\usepackage[utf8]{inputenc} % make weird characters work
\usepackage{graphicx}
%\usepackage[]{algorithm2e}
\usepackage{caption}
\usepackage{float}

\usepackage[english,serbian]{babel}
%\usepackage[english,serbianc]{babel} %ukljuciti babel sa ovim opcijama, umesto gornjim, ukoliko se koristi cirilica

\usepackage[unicode]{hyperref}
\hypersetup{colorlinks,citecolor=green,filecolor=green,linkcolor=blue,urlcolor=blue}

\usepackage{listings}

%\newtheorem{primer}{Пример}[section] %ćirilični primer
\newtheorem{primer}{Primer}[section]

\definecolor{mygreen}{rgb}{0,0.6,0}
\definecolor{mygray}{rgb}{0.5,0.5,0.5}
\definecolor{mymauve}{rgb}{0.58,0,0.82}

\lstset{ 
  backgroundcolor=\color{white},   % choose the background color; you must add \usepackage{color} or \usepackage{xcolor}; should come as last argument
  basicstyle=\scriptsize\ttfamily,        % the size of the fonts that are used for the code
  breakatwhitespace=false,         % sets if automatic breaks should only happen at whitespace
  breaklines=true,                 % sets automatic line breaking
  captionpos=b,                    % sets the caption-position to bottom
  commentstyle=\color{mygreen},    % comment style
  deletekeywords={...},            % if you want to delete keywords from the given language
  escapeinside={\%*}{*)},          % if you want to add LaTeX within your code
  extendedchars=true,              % lets you use non-ASCII characters; for 8-bits encodings only, does not work with UTF-8
  firstnumber=1000,                % start line enumeration with line 1000
  frame=single,	                   % adds a frame around the code
  keepspaces=true,                 % keeps spaces in text, useful for keeping indentation of code (possibly needs columns=flexible)
  keywordstyle=\color{blue},       % keyword style
  language=Python,                 % the language of the code
  morekeywords={*,...},            % if you want to add more keywords to the set
  numbers=left,                    % where to put the line-numbers; possible values are (none, left, right)
  numbersep=5pt,                   % how far the line-numbers are from the code
  numberstyle=\tiny\color{mygray}, % the style that is used for the line-numbers
  rulecolor=\color{black},         % if not set, the frame-color may be changed on line-breaks within not-black text (e.g. comments (green here))
  showspaces=false,                % show spaces everywhere adding particular underscores; it overrides 'showstringspaces'
  showstringspaces=false,          % underline spaces within strings only
  showtabs=false,                  % show tabs within strings adding particular underscores
  stepnumber=2,                    % the step between two line-numbers. If it's 1, each line will be numbered
  stringstyle=\color{mymauve},     % string literal style
  tabsize=2,	                   % sets default tabsize to 2 spaces
  title=\lstname                   % show the filename of files included with \lstinputlisting; also try caption instead of title
}

\begin{document}

\title{Fazi neuronska mreža\\ \small{Seminarski rad u okviru kursa\\Računarska inteligencija\\ Matematički fakultet}}

\author{Katarina Savičić, Bojana Ristanović\\ mi16261@alas.matf.bg.ac.rs, mi16045@alas.matf.bg.ac.rs}

\maketitle

\abstract{
U ovom radu će biti predstavljena fazi neuronska mreža. Konkretnije, ovaj metod je korišćen kao metod klasifikacije podataka,
koji u trening fazi koristi min-max algoritam, a u test fazi koristi k najbližih suseda. Nakon opisa algoritma, prikaza je njegova
praktična primena. Na kraju, rezultati su upoređeni sa autorima rada \emph{Understanding Fuzzy Neural Network using code and animation}.
\\
\\
\textbf{Ključne reči:} fazi logika, neuronske mreže, min-max, knn, klasifikacija.
}
\tableofcontents

\newpage

\section{Uvod}
\label{sec:uvod}

Fazi neuronska mreža (eng. \emph{Fuzzy neural network}) je sistem za učenje koji pronalazi parametre fazi sistema koristeći tehnike aproksimacije neuronskih mreža. To je hibridni inteligentni sistem koji kombinuje tehnike rasuđivanja fazi logike sa tehnikama učenja neuronskih mreža.\cite{fnn}

Fazi min-max klasfikator (eng. \emph{The fuzzy min–max (FMM)}) je sistem koji formira hiperboksove za klasifikaciju i predviđanje. U ovom radu je pokušana modifikacija FMM-a korišćenjem algoritma k najbližih suseda (eng. \emph{k-nearest neighbors algorithm (k-NN)}) u procesu predviđanja klasa prosleđenih podataka.

\section{Fazi logika}
\label{neuronskemreze}


\section{Neuronske Mreže}
\label{neuronskemreze}


\section{Fazi min-max klasifikator}
\label{neuronskemreze}


\section{Rezultati}
\label{neuronskemreze}


\section{Zaključak}
\label{sec:zakljucak}

\addcontentsline{toc}{section}{Literatura}
\appendix
\bibliography{seminarski} 
\bibliographystyle{plain}

\newpage
\appendix


\end{document}
